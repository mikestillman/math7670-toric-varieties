\documentclass{exam}
\pagestyle{head}
\firstpageheader{Math 7670 -- Toric Varieties -- Suggested Problems \# 2 -- Spring 2019}{}{}

\usepackage{amsmath,amsthm,amssymb}
 
\newcommand{\N}{\mathbb{N}}
\newcommand{\Z}{\mathbb{Z}}
\newcommand{\rr}{\mathbb{R}}
\newcommand{\cc}{\mathbb{C}}
\newcommand{\zz}{\mathbb{Z}}
\newcommand{\ff}{\mathbb{F}}
 
\newenvironment{theorem}[2][Theorem]{\begin{trivlist}
\item[\hskip \labelsep {\bfseries #1}\hskip \labelsep {\bfseries #2.}]}{\end{trivlist}}
\newenvironment{lemma}[2][Lemma]{\begin{trivlist}
\item[\hskip \labelsep {\bfseries #1}\hskip \labelsep {\bfseries #2.}]}{\end{trivlist}}
\newenvironment{exercise}[2][Exercise]{\begin{trivlist}
\item[\hskip \labelsep {\bfseries #1}\hskip \labelsep {\bfseries #2.}]}{\end{trivlist}}
\newenvironment{problem}[2][Problem]{\begin{trivlist}
\item[\hskip \labelsep {\bfseries #1}\hskip \labelsep {\bfseries #2.}]}{\end{trivlist}}
\newenvironment{question}[2][Question]{\begin{trivlist}
\item[\hskip \labelsep {\bfseries #1}\hskip \labelsep {\bfseries #2.}]}{\end{trivlist}}
\newenvironment{corollary}[2][Corollary]{\begin{trivlist}
\item[\hskip \labelsep {\bfseries #1}\hskip \labelsep {\bfseries #2.}]}{\end{trivlist}}

\begin{document}

We will discuss these problems in class on February 21.  This time (as
opposed to what we did on the first set of suggested problems), I will
not present any of the solutions.  Participants will!

\begin{questions}

  \question Suppose that $\sigma$ and $\tau$ are polyhedral cones in $\rr^n$.
  Show that
  \begin{parts}
  \part Show that $\sigma^\vee \cap \tau^\vee = (\sigma + \tau)^\vee$.
  \part Show that $\sigma^\vee + \tau^\vee = (\sigma \cap \tau)^\vee$.
  \end{parts}

  \question Suppose that $\sigma$ is not pointed.  What is $X_\sigma$ in this case?

  \question Show that if $\sigma$ and $\sigma'$ are cones in $N$, $N'$ (respectively), then
    there is a natural isomorphism $X_{\sigma \times \sigma'} = X_\sigma \times X_{\sigma'}$.
  
  \question Recall that an irreducible affine variety $X$ is called {\bf\it normal} if its
  coordinate ring $A(X)$ is integrally closed in its fraction field.  Prove that if $\sigma$
  is a pointed cone in $N$, then the affine toric variety $X_\sigma$ is normal.

  \question Suppose that $S \subset T \subset M = \zz^n$ are sub semigroups.
  Show that the natural induced map $Spec(\cc[T]) \longrightarrow Spec(\cc[S])$
  is birational if and only if $S$ and $T$ generate the same subgroup of $M$.

  \question Consider an inclusion of lattices $N \subset N' \subset \rr^n$, of index $\ell$
  and suppose that $\sigma$ is a pointed rational polyhedral cone in $\rr^n$.
  \begin{parts}
    \part Describe the corresponding morphism on toric varieties.
    \part Is this morphism finite?  If so, what is its degree?
    \end{parts}
  
  \question Suppose that $X = X_\sigma$ is an affine toric variety.  Is the
  singular locus also an affine toric variety?  Identify the singular locus of $X$.
  (e.g. in terms of orbits, or in terms of $V(\tau)$'s).  What is the maximum
  dimension of the singular locus?  In particular, can it be of codimension 1 in $X$?


  \question Find the ideal of the affine toric variety corresponding to the cone
  of $3 \times 3$ magic squares, from the first set of suggested problems.  What is the
  singular locus of this variety?  What are the ideals of the codimension one
  subvarieties of the form $V(\tau)$?

  \question Work through (in gory detail) the proof that $X_\Sigma$ is well defined (i.e. that the
  glueing axioms all hold).  State clearly what lemma's about cones (and rational cones) you need.
\end{questions}

\end{document}

  \question Let $\Sigma \subset N_\rr = \rr^2$ be a complete fan in the plane, and
  suppose that the corresponding toric variety is smooth.  Show that there is a fan $\Sigma'$
  such that (a) $\Sigma'$ has either 3 or 4 rays, and (b) $\Sigma$ is a refinement of $\Sigma'$.

  \question Let $\ff_2$ be the toric variety corresponding to the fan with rays $(1,0)$, $(0,1)$, $(-1,2)$ and $(0,-1)$.
  Find all of the $T$-invariant subvarieties of $\ff_2$.

