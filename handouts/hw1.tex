\documentclass{exam}
\pagestyle{head}
\firstpageheader{Math 7670 -- Toric Varieties -- Suggested Problems \# 1 -- Spring 2019}{}{}

\usepackage{amsmath,amsthm,amssymb}
 
\newcommand{\N}{\mathbb{N}}
\newcommand{\Z}{\mathbb{Z}}
\newcommand{\rr}{\mathbb{R}}
 
\newenvironment{theorem}[2][Theorem]{\begin{trivlist}
\item[\hskip \labelsep {\bfseries #1}\hskip \labelsep {\bfseries #2.}]}{\end{trivlist}}
\newenvironment{lemma}[2][Lemma]{\begin{trivlist}
\item[\hskip \labelsep {\bfseries #1}\hskip \labelsep {\bfseries #2.}]}{\end{trivlist}}
\newenvironment{exercise}[2][Exercise]{\begin{trivlist}
\item[\hskip \labelsep {\bfseries #1}\hskip \labelsep {\bfseries #2.}]}{\end{trivlist}}
\newenvironment{problem}[2][Problem]{\begin{trivlist}
\item[\hskip \labelsep {\bfseries #1}\hskip \labelsep {\bfseries #2.}]}{\end{trivlist}}
\newenvironment{question}[2][Question]{\begin{trivlist}
\item[\hskip \labelsep {\bfseries #1}\hskip \labelsep {\bfseries #2.}]}{\end{trivlist}}
\newenvironment{corollary}[2][Corollary]{\begin{trivlist}
\item[\hskip \labelsep {\bfseries #1}\hskip \labelsep {\bfseries #2.}]}{\end{trivlist}}

\begin{document}

We will discuss these problems in class on February 5.

\begin{questions}
  
  \question (Pointed cones and lineality spaces)
  Let $\sigma \subset \rr^m$ be a polyhedral cone.  The {\bf lineality space} of $\sigma$ is
  the largest subspace contained in $\sigma$. The cone $\sigma$ is called {\bf pointed} (or {\bf strongly convex}) if
  its lineality space is zero.

  \begin{parts}
    \part Show that the following are equivalent.
    \begin{itemize}
    \item $\sigma \cap (- \sigma) = 0$.
    \item $\sigma$ contains no non-zero linear subspace.
    \item There is a $u \in \sigma^\vee$ with $\sigma \cap u^\perp = \{0\}$.
      \item $\sigma^\vee$ spans $V$.
    \end{itemize}
    \part Show that $\sigma$ can be written as the sum of a linear subspace and a pointed cone.
    In fact, 
       \[ \sigma = L + \sigma_1, \]
    where $L = \sigma \cap (-\sigma)$ is the lineality space, and $\sigma_1 = \sigma \cap L^\perp$.
  \end{parts}

  \question Prove the following: If $\sigma^\vee$ is minimally generated by $\{ u_1, \ldots, u_s\} \subset V^*$, show that
  $\tau_i := \sigma \cap u_i^\perp$ is a facet of $\sigma$.

  \question Prove the Fourier-Motzkin elimination theorem, by mimicking the example we did after we stated it (see the notes for the exact statement).
  
  \question {\bf Macaulay2}.
  Either install Macaulay2 on your computer or laptop, or learn how to use the online version at web.macaulay2.com (e.g. go
  through one or several of the tutorials that you will find there).
  Try the following example:
\begin{verbatim}
  needsPackage "FourierMotzkin"
  viewHelp "FourierMotzkin"
  A = transpose matrix{{1,1,1,1}, {1,2,3,4}, {1,3,5,7}, {2,4,8,16}, {3,9,27,81}}
  (Bc, Bh) = fourierMotzkin A
  (Cc, Ch) = fourierMotzkin Bc
  (transpose Bc) * Cc
  \end{verbatim}
  \begin{parts}
    \part From the output of this snippet of code, find irredundant generators for $\sigma = vcone(A)$ and for $\sigma^\vee$.
    \part Find the faces of $\sigma$ and $\sigma^\vee$, and determine which is dual to which.

  \medskip\noindent Now try this:
  \begin{verbatim}
  needsPackage "Polyhedra"
  C = coneFromVData A
  rays C
  faces C
  halfspaces C
  C' = dualCone  C
  rays C'
  hilbertBasis C
  matrix {hilbertBasis C}
  \end{verbatim}

  \part Does the Polyhedra package represent inequalities using rows, or using columns? (e.g. for 
  {\tt halfspaces} and for {\tt dualCone}, are the generators the rows or the columns?)
  \end{parts}
  
  \question {\bf The cone of magic squares.}

  Consider the cone $\sigma \subset \rr^9 = \rr^{3 \times 3}$
 consisting of all $3 \times 3$ matrices with nonnegative real
 coefficients, whose row, column, and diagonal sums are all equal to each other.

  You may wish to use Macaulay2 for this exercise, but also try to do it by hand first.
  \begin{parts}
    \part Find a matrix $A \in \rr^{N \times 9}$ such that $\sigma = hcone(A)$.
    \part Is $\sigma$ a pointed cone?  A rational cone?
    \part Find an irredundant set of generators for $\sigma$.
    \part Find the Hilbert basis for $\sigma$.
    \end{parts}

\end{questions}

\end{document}
